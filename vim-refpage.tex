\documentclass[twocolumn,a4paper]{article}

\usepackage{tgheros}
\usepackage{fourier}
\usepackage{lmodern}
\renewcommand*\familydefault{\sfdefault}
%\renewcommand*\ttdefault{lcmtt}
\usepackage[T1]{fontenc}
\usepackage{anyfontsize}
\usepackage{calc}
\usepackage{geometry}
\usepackage[table]{xcolor}
\usepackage{array}
\usepackage{multirow}
\usepackage{tabulary}
\usepackage{setspace}
\geometry{margin=0pt}

%%%% DEFINE \setbraceratio AND \resetbraceratio MACROS
\usepackage{etoolbox}
\usepackage{multido}

\makeatletter
% default brace filler
\def\@@bfil{\leaders \vrule \@height \ht\z@ \@depth \z@ \hfill}
% left leader filler
\def\@bLfil{\@@bfil}
% right leader filler
\def\@bRfil{\@@bfil}
% reset to default braces
\def\resetbraceratio{\gdef\@bLfil{\@@bfil}\gdef\@bRfil{\@@bfil}}
% \setbraceratio{<left>}{<right>}
\def\setbraceratio#1#2{
	% clear left filler
	\let\@bLfil\relax
	% increase left ratio
	\multido{\iA=1+1}{#1}{\gappto\@bLfil{\@@bfil}}
	% clear right filler
	\let\@bRfil\relax
	% increase right ratio
	\multido{\iA=1+1}{#2}{\gappto\@bRfil{\@@bfil}}
}
\def\upbracefill{$\m@th\setbox\z@\hbox{$\braceld$}\bracelu\@bLfil\bracerd\braceld\@bRfil\braceru$}
\def\downbracefill{$\m@th\setbox\z@\hbox{$\braceld$}\braceld\@bLfil\braceru\bracelu\@bRfil\bracerd$}
\makeatother
%%%%% END OF MACROS

\def\w#1{\textcolor{white}{#1}}
\def\r#1{\textcolor{red}{#1}}
\def\b#1{\textcolor{blue}{#1}}
\def\g#1{\textcolor{gray}{#1}}

\setlength{\extrarowheight}{2pt}

% Tabular environment with alternating row colours
% Params: (as per a normal tabular)
\newenvironment{coltabular}[2][c]
	{\rowcolors{1}{cyan!35}{white}\begin{tabular}[#1]{#2}}
	{\end{tabular}\rowcolors{1}{white}{white}}

% Tabular environment for entering <cmd> <description> pairs.
% Params:
% 	1 - optional (default: c) table position
% 	2 - required - number of columns
% TODO: the description columns should have \descstyle set, but this makes the
%       text jump down a line
% TODO: why do we need so much extrarowheight in these tables?
% TODO: the top of the first line of the description should align with the top
%       of the first line of the command. At the moment the bottoms align.
\newenvironment{cmdtable}[2][c]
	%{\rowcolors{1}{gray!50}{white}\arrayrulecolor{white}\begin{tabular}[#1]{*#2{|>{\setsize8\cmdstyle}c>{\descstyle}X|}}}
	{\rowcolors{1}{gray!50}{white}%
	 \setlength{\extrarowheight}{6pt}%
	 \arrayrulecolor{white}%
	 \tabulary{\linewidth}{*#2{|>{\setsize8\cmdstyle}c>{\setsize4\setstretch{0.75}}L|}}}
	{\endtabulary\rowcolors{1}{white}{white}}

\parindent0pt


\def\fbbox#1{\mbox{\rlap{\fbox{\phantom{#1}}}{\kern3.5pt{#1}\kern3.5pt}}}
\def\dl#1{
	\raisebox{-0.3\baselineskip}
	[-0.3\baselineskip]
	[-0.3\baselineskip]{#1}
}

\def\nf{\normalfont}

\def\inlinecmd#1{\colorbox{gray!50}{\kern-3pt{\ttfamily\bfseries\color{black}#1}\kern-3pt}}
%\def\inlinecmd#1{{\color{black}\tt\bfseries#1}}
\def\cursorhl#1{\colorbox{black}{\kern-3pt{\color{white}#1}\kern-3pt}}
\def\titlestyle{\sf\sc\color{black}}
\def\parstyle{\rm\color{black}}
\def\cmdstyle{\bfseries\ttfamily\color{black}}
\def\egcmdstyle{\itshape\ttfamily\color{red}}
\def\descstyle{\sf\color{black}}

\def\usetextobjectsbox{%
	\large\tt(\fbbox{\parbox[t]{\widthof{use text-objects}+14pt}{
		use
		\fbbox{\parbox[t]{\widthof{text-objects}+7pt}{%
			\fbbox{\parbox[t]{\widthof{text}}{
				tex\cursorhl t\\[-0.3\baselineskip]
				\centering\dl{\colorbox{white}{\b{iw}}}%
			}}-objects\\[0.1\baselineskip]%
			\centering\dl{\colorbox{white}{\b{iW}}}%
		}}\\[0.1\baselineskip]%
		\dl{\colorbox{white}{\b{i(}}}
	}}%
	)
}

%%% Tag object with help information %%%
% \taghelp{object}{help-cmd}
\newlength\helphgt
\newlength\helpdep
\settoheight{\helphgt}{\tiny\strut}
\settodepth{\helpdep}{\tiny\strut}
\def\taghelp#1#2{%
	{#1}%
	\llap{\raisebox{\heightof{#1}+\helpdep}{\tiny\egcmdstyle{:h #2}\kern2ex}}%
}
%%%%%%%%%%

%%%%% SIZES MACROS
\newcounter{sizecounter}
\def\setsize#1{%
	\ifnum#1<1%
		\global\tiny%
	\else\ifcase#1%
		\relax\or%                                              0
		\setcounter{sizecounter}{#1}\global\tiny\or%            1
		\setcounter{sizecounter}{#1}\global\scriptsize\or%      2
		\setcounter{sizecounter}{#1}\global\footnotesize\or%    3
		\setcounter{sizecounter}{#1}\global\small\or%           4
		\setcounter{sizecounter}{#1}\global\normalsize\or%      5
		\setcounter{sizecounter}{#1}\global\large\or%           6
		\setcounter{sizecounter}{#1}\global\Large\or%           7
		\setcounter{sizecounter}{#1}\global\LARGE\or%           8
		\setcounter{sizecounter}{#1}\global\huge\or%            9
		\setcounter{sizecounter}{#1}\global\Huge\else%         10
		\global\Huge%                                         >10
	\fi\fi%
}
\makeatletter
\def\incsize{\@ifnextchar[{\@incsize}{\@incsize[1]}}
\def\decsize{\@ifnextchar[{\@decsize}{\@decsize[1]}}
\def\@incsize[#1]#2{%
	\addtocounter{sizecounter}{#1}%
	\setsize{\value{sizecounter}}%
	#2%
	\addtocounter{sizecounter}{-#1}%
	\setsize{\value{sizecounter}}%
}
\def\@decsize[#1]#2{%
	\addtocounter{sizecounter}{-#1}%
	\setsize{\value{sizecounter}}%
	#2%
	\addtocounter{sizecounter}{#1}%
	\setsize{\value{sizecounter}}%
}
\makeatother
%%%%%%%

\def\desccmd#1#2{\decsize[5]{%
	\begin{tabular}{@{}c@{}}
		\incsize[5]{\egcmdstyle #1}\\
		\descstyle #2
	\end{tabular}%
}}

\begin{document}


{\setsize{8}\tt
	\taghelp{[\r{operator}]}{operator}%
	\g{[count]}\taghelp{[\b{motion}]}{navigation}}

\begin{tabular}[t]{@{}p{0.25\linewidth}
	@{\hspace{0.05\linewidth}}p{0.35\linewidth}
	@{\hspace{0.05\linewidth}}p{0.3\linewidth}@{}}
	\downbracefill &  \omit\span\omit\setbraceratio 55\downbracefill\\
	\setsize7\begin{tabular}[t]{@{}p{\linewidth}@{}}
    \centering\desccmd{d}{delete/cut}\\
    \centering\desccmd{y}{yank/copy}\\
	\centering\desccmd{c}{change}\\
  \end{tabular}
  & \setsize2\parstyle Any motion can follow an operator. Marks and searches count 
  as motions too! \inlinecmd{d/foo} will delete from the cursor to the next 
  instance of ``foo''. \inlinecmd{y3fi} will yank from the cursor to the 3rd 
  ``i'' on the line after it. Counts can also come before operators: 
  \inlinecmd{5dd} will delete five lines.    
  & \setsize2\begin{coltabular}[t]{>{\cmdstyle}c>{\descstyle}l}
	w & word \\
	W & WORD \\
	s & sentence \\\relax
	[, ] & [ ] block \\
	(, ) & ( ) block \\
	<, > & < > block \\
	t  & XML/HTML tag \\
	\{, \} & \{ \} block \\
	", ' & quoted string
\end{coltabular} \\[-2ex]
%TODO: use tabulary for this
  \setsize2\centering\begin{tabular}{@{}cc@{}}
	  \desccmd{gU}{make\\ upper-case} & \desccmd{Y}{swap case}\\[2ex]
 	\desccmd{<}{shift left} & \desccmd{=}{indent}\\
 \end{tabular} & 
 \multicolumn{2}{c}{\raisebox{-4ex}{%
	 \taghelp{\usetextobjectsbox}{text-objects}
 }}
\end{tabular}


\vspace{3ex}
{\setsize{3}\arrayrulecolor{gray!40}\tt\begin{tabular}{|*5{>{\cmdstyle}c|}}
	\hline
	\rowcolor{cyan!35}\multicolumn5{|l|}{\titlestyle Searching}\\\hline
	\titlestyle Prev & \titlestyle Next 
		&\titlestyle Forward &\titlestyle Backward & \titlestyle Matches\\\hline
	\multirow 2*N & \multirow 2*n &
		/\egcmdstyle foo & ?\egcmdstyle foo & \egcmdstyle foo\\
		\cline{3-5}
		&& * & \# & \decsize[1]{\descstyle word under cursor}
	\\
	\hline
	\multirow 2*; & \multirow 2*, &
	t\egcmdstyle x & T\egcmdstyle x & \decsize[1]{\descstyle upto}
			\egcmdstyle x\\
			\cline{3-5}
			&& f\egcmdstyle x & F\egcmdstyle x & \decsize[1]{\descstyle find} 
			\egcmdstyle x 
	\\\hline
\end{tabular}}

% marks table
% TODO: work out the best way of putting the descriptions on multiple lines.
%       can we specify a width, put them in boxes and have everything calculated automatically?
\vspace{3ex}
\taghelp{
	\arrayrulecolor{white}
	\begin{cmdtable}{3}
		m\egcmdstyle{m} & create mark {\egcmdstyle m} in file &
		m\egcmdstyle{M} & set mark {\egcmdstyle M} across files &
		\textquotesingle[ & jump to first char of just-changed text \\
		\textquotesingle\egcmdstyle{m} & jump to first char of line containing \egcmdstyle m &
		\`\egcmdstyle m & jump to exact char of \egcmdstyle m &
		\textquotesingle\textquotesingle & jump to previous jump point
	\end{cmdtable}
}{mark-motions}
\end{document}
