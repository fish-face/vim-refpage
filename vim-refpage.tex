\documentclass[12pt,twocolumn,a4paper]{article}

\usepackage{calc}
\usepackage{geometry}
\usepackage[table]{xcolor}
\usepackage{fourier}
\geometry{margin=0pt}

\usepackage{etoolbox}
\usepackage{multido}

\makeatletter
\def\@@bfil{\leaders \vrule \@height \ht\z@ \@depth \z@ \hfill}% default brace filler
\def\@bLfil{\@@bfil}% left leader filler
\def\@bRfil{\@@bfil}% right leader filler
\def\resetbraceratio{\gdef\@bLfil{\@@bfil}\gdef\@bRfil{\@@bfil}}% reset to default braces
\def\setbraceratio#1#2{% \setbraceratio{<left>}{<right>}
  \let\@bLfil\relax% clear left filler
    \multido{\iA=1+1}{#1}{\gappto\@bLfil{\@@bfil}}% increase left ratio
      \let\@bRfil\relax% clear right filler
        \multido{\iA=1+1}{#2}{\gappto\@bRfil{\@@bfil}}% increase right ratio
	}
	\def\upbracefill{$\m@th\setbox\z@\hbox{$\braceld$}\bracelu\@bLfil\bracerd\braceld\@bRfil\braceru$}
	\def\downbracefill{$\m@th\setbox\z@\hbox{$\braceld$}\braceld\@bLfil\braceru\bracelu\@bRfil\bracerd$}
	\makeatother

\def\w#1{\textcolor{white}{#1}}
\def\r#1{\textcolor{red}{#1}}
\def\b#1{\textcolor{blue}{#1}}
\def\g#1{\textcolor{gray}{#1}}

\newenvironment{coltabular}[2][c]
	{\rowcolors{1}{cyan!35}{white}\begin{tabular}[#1]{#2}}
	{\end{tabular}\rowcolors{1}{white}{white}}

\parindent0pt

\def\cd#1#2{\begin{tabular}[t]{@{}c@{}} \parbox{\linewidth}{\centering 
  #1}\\[-0.5ex] \parbox{\linewidth}{\centering #2}\\ \end{tabular}}

\def\fbbox#1{\mbox{\rlap{\fbox{\phantom{#1}}}{\kern3.5pt{#1}\kern3.5pt}}}
\def\dl#1{
	\raisebox{-0.3\baselineskip}
	[-0.3\baselineskip]
	[-0.3\baselineskip]{#1}
}

\def\inlinecmd#1{\colorbox{gray}{\kern-3pt{\color{black}#1}\kern-3pt}}
\def\cursorhl#1{\colorbox{black}{\kern-3pt{\color{white}#1}\kern-3pt}}

\def\usetextobjectsbox{%
	\large\tt(\fbbox{\parbox[t]{\widthof{use text-objects}+14pt}{
		use
		\fbbox{\parbox[t]{\widthof{text-objects}+7pt}{%
			\fbbox{\parbox[t]{\widthof{text}}{
				tex\cursorhl t\\[-0.3\baselineskip]
				\centering\dl{\colorbox{white}{\b{iw}}}%
			}}-objects\\[0.1\baselineskip]%
			\centering\dl{\colorbox{white}{\b{iW}}}%
		}}\\[0.1\baselineskip]%
		\dl{\colorbox{white}{\b{i(}}}
	}}%
	)
}

\newlength\helphgt
\newlength\helpdep
\settoheight{\helphgt}{\tiny\strut}
\settodepth{\helpdep}{\tiny\strut}
\def\taghelp#1#2{%
	{#1}%
	\llap{\raisebox{\heightof{#1}+\helpdep}{\tt\tiny\r{:h #2}\kern2ex}}%
}
\begin{document}

{\LARGE\tt
	\taghelp{[\r{operator}]}{operator}%
	\g{[count]}\taghelp{[\b{motion}]}{motion}
}

\begin{tabular}[t]{@{}p{0.25\linewidth}
	@{\hspace{0.05\linewidth}}p{0.35\linewidth}
	@{\hspace{0.05\linewidth}}p{0.3\linewidth}@{}}
	\downbracefill &  \omit\span\omit\setbraceratio 53\downbracefill\\
	\begin{tabular}[t]{@{}c@{}}
    \cd{\large\tt\r d}{\scriptsize delete/cut}\\
    \cd{\large\tt\r y}{\scriptsize yank/copy}\\
	\cd{\large\tt\r c}{\scriptsize change}
  \end{tabular}
  & \scriptsize Any motion can follow an operator. Marks and searches count 
  as motions too! \inlinecmd{\tt d/foo} will delete from the cursor to the next 
  instance of ``foo''. \inlinecmd{\tt y3fi} will yank from the cursor to the 3rd 
  ``i'' on the line after it. Counts can also come before operators: 
  \inlinecmd{\tt 5dd} will delete five lines.    
  & \scriptsize\begin{coltabular}[t]{cl}
	\tt w & word \\
	\tt W & WORD \\
	\tt s & sentence \\
	\tt [, ] & [ ] block \\
	\tt (, ) & ( ) block \\
	\tt <, > & < > block \\
	\tt t  & XML/HTML tag \\
	\tt \{, \} & \{ \} block \\
	\tt ", ' & quoted string
  \end{coltabular} \\[-2.5ex]
  \centering\begin{tabular}[c]{@{}p{0.4\linewidth}
	@{\hspace{0.1\linewidth}}p{0.4\linewidth}@{}}
     \cd{\footnotesize\tt\r{gU}}{\tiny make\linebreak uppercase} &
     \cd{\footnotesize\tt\r Y}{\tiny swap case}\\
     \cd{\footnotesize\tt\r <}{\tiny shift left} &
	 \cd{\footnotesize\tt\r =}{\tiny indent}
 \end{tabular} & 
 \multicolumn{2}{c}{\raisebox{-2ex}{%
	 \taghelp{\usetextobjectsbox}{text-objects}
 }}
 
\end{tabular}


\end{document}
